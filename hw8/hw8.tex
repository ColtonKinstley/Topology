\documentclass[11pt, oneside]{article}   	% use "amsart" instead of "article" for AMSLaTeX format
\usepackage{geometry}                		% See geometry.pdf to learn the layout options. There are lots.
\geometry{letterpaper}                   		% ... or a4paper or a5paper or ... 
%\geometry{landscape}                		% Activate for rotated page geometry
%\usepackage[parfill]{parskip}    		% Activate to begin paragraphs with an empty line rather than an indent
\usepackage{graphicx}				% Use pdf, png, jpg, or eps§ with pdflatex; use eps in DVI mode

								% TeX will automatically convert eps --> pdf in pdflatex		
\usepackage{amssymb}

\usepackage{amsmath}
\usepackage[mathscr]{euscript}
\usepackage[shortlabels]{enumitem}

\newcommand{\R}{\mathbb{R}}
\newcommand{\Z}{\mathbb{Z}}
\newcommand{\Q}{\mathbb{Q}}
\newcommand{\C}{\mathbb{C}}
\newcommand{\F}{\mathbb{F}}
\newcommand{\bb}[1]{\mathbb{#1}}
\newcommand{\scr}[1]{\mathscr{#1}}
\newcommand{\tand}{\text{ and }}
\newcommand{\tor}{\text{ or }}
\newcommand{\twhere}{\text{ where }}
\newcommand{\tfor}{\text{ for }}
\newcommand{\st}{\mid}
\newcommand{\qed}{\begin{center}
$\square$
\end{center}}
\newcommand{\nullset}{\varnothing}
\newcommand{\set}[1]{\{ #1 \}}

%SetFonts

%SetFonts


\title{Selected Exercises \S23 \& \S28}
\author{Colton Kinstley}
%\date{}							% Activate to display a given date or no date

\begin{document}
\maketitle

\section*{Question \S23-7}
\paragraph{}

Is the space $\R_\ell$ connected? Justify your answer.

\subsection*{Solution}
\paragraph{}

No, $\R_\ell$ is not connected. To see this consider the open sets formed by fixing $b$ in $\R_\ell$

\begin{eqnarray*}
U_1 = \bigcup_{x<b}[x,b) = (-\infty, b) \\
U_2 = \bigcup_{b<x}[b,x) = [b,\infty).
\end{eqnarray*}

The sets $U_1 \tand U_2$ clearly form a separation of $\R_\ell$.

\section*{Question \S23-11}
\paragraph{}
Let $p:X \to Y$ be a quotient mp. Show that if each set $p^{-1}(\set{y})$ is connected, and if $Y$ is connected, then $X$ is connected.

\subsection*{Solution}
\paragraph{}

Suppose $X$ is not connected, let $A \tand B$ be a separation of $X$. We will show that $A \tand B$ are saturated with respect to $p$. 

Given any $y \in Y$ the set $p^{-1}(\set{y})$ must lie entirely within $A \tor B$ since $p^{-1}(\set{y})$ is connected. Thus $A$ must contain any set $p^{-1}(\set{y})$ that it intersects. Likewise for $B$. Further, note that since $p$ is surjective $p(A) \cup p(B) = Y$ and $p(A) \cup p(B) = \nullset$, otherwise a point in the intersection would have pre-image lying in both $A \tand B$. Since $p$ takes saturated open sets of $X$ to open sets in $Y$, $p(A) \tand p(B)$ are open and thus form a separation of $Y$, a contradiction.

\section*{Question \S28-3}
\paragraph{}

Let $X$ be limit point compact.

\begin{enumerate}[(a)]
\item
If $f:X \to Y$ is continuous, does it follow that $f(X)$ is limit point compact?
\item
If $A$ is a closed subset of $X$, does it follow that $A$ is limit point compact?
\end{enumerate}

\subsection*{Solution (a)}
\paragraph{}
 
Yes, $f(X)$ is limit point compact. Suppose $A \subset f(X)$ and $A$ has infinitely man points. Then $f^{-1}(A)$ has infinitely many points and so has a limit point in $X$, say $x$. Then there exists a net in $f^{-1}(A)$, say $\set{x_\lambda}_{\lambda \in \Lambda}$ such that $x_\lambda \to x$. By continuity of $f$, $f(x_\lambda \to f(x)$, thus $f(x)$ is a limit point of $A$.

\subsection*{Solution (b)}
\paragraph{}

Yes, $A$ is limit point compact. If $A$ is closed then $A$ contains all of it's limit points. Given $B$, an infinite subset of $A$, $B$ has a limit point in $X$, being an infinite subset of $X$. But since $B \subset A$ this limit point must lie within $A$. Thus $A$ is limit point compact.


 
 
 
 
 
 
 
 
 























































\end{document}
