\documentclass[11pt, oneside]{article}   	% use "amsart" instead of "article" for AMSLaTeX format
\usepackage{geometry}                		% See geometry.pdf to learn the layout options. There are lots.
\geometry{letterpaper}                   		% ... or a4paper or a5paper or ... 
%\geometry{landscape}                		% Activate for rotated page geometry
%\usepackage[parfill]{parskip}    		% Activate to begin paragraphs with an empty line rather than an indent
\usepackage{graphicx}				% Use pdf, png, jpg, or eps§ with pdflatex; use eps in DVI mode

								% TeX will automatically convert eps --> pdf in pdflatex		
\usepackage{amssymb}

\usepackage{amsmath}
\usepackage[mathscr]{euscript}


%SetFonts

%SetFonts


\title{Characterization of the Finite Complement Topology on the Real Line}
\author{Colton Kinstley}
%\date{}							% Activate to display a given date or no date

\begin{document}
\maketitle



\section*{Abstract}
\paragraph{}

A proof that all open sets in the finitie complement copology on the real lie can be written as the finite union $(- \infty , a_1) \cup (a_1,a_2) \cup \dots \cup (a_{n-1}, a_n) \cup (a_n, \infty)$.


\section*{Definitions}
\paragraph{}

Given a set $X$ let $\mathscr{T}_f$ be the collection of all subsets $U$ of $X$ such that $X \setminus U$ is either all of $X$ or is finite. That is,

\[
\mathscr{T}_f = \{ U \subset X | U \setminus X = X \text{ or } U \setminus X \text{ is finite} \}.
\]




\section*{A Charictarization of $\mathscr{T}_f$}
\paragraph{}

Given any open set $U$ in $(\mathbb{R},\mathscr{T}_f)$, $U$ is either all of $\mathbb{R}$ or can be written in the form

\[
U = (- \infty , a_1) \cup (a_1,a_2) \cup \dots \cup (a_{n-1}, a_n) \cup (a_n, \infty)
\]
with $a_i \in \mathbb{R}$ and $ i \in \mathbb{N}$.

\subsection*{Proof}
\paragraph{}

First we claim that given any $U$ open in $\mathbb{R}$ there must exist some $n \in \mathbb{N}$ such that
\[
\mathbb{R} \setminus U = \{x_i\}_{i=1}^n \text{ with } x_i \in \mathbb{R}.
\]

If not then either the RHS is empty and the hypotheses holds with $U = $. 

































\end{document}