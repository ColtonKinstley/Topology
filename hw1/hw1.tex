\documentclass[11pt, oneside]{article}   	% use "amsart" instead of "article" for AMSLaTeX format
\usepackage{geometry}                		% See geometry.pdf to learn the layout options. There are lots.
\geometry{letterpaper}                   		% ... or a4paper or a5paper or ... 
%\geometry{landscape}                		% Activate for rotated page geometry
%\usepackage[parfill]{parskip}    		% Activate to begin paragraphs with an empty line rather than an indent
\usepackage{graphicx}				% Use pdf, png, jpg, or eps§ with pdflatex; use eps in DVI mode

								% TeX will automatically convert eps --> pdf in pdflatex		
\usepackage{amssymb}

\usepackage{amsmath}
\usepackage[mathscr]{euscript}

\newcommand{\R}{\mathbb{R}}
\newcommand{\bb}[1]{\mathbb{#1}}
\newcommand{\scr}[1]{\mathscr{#1}}
\newcommand{\tand}{\text{ and }}
\newcommand{\tor}{\text{ or }}
\newcommand{\twhere}{\text{ where }}
\newcommand{\tfor}{\text{ for }}
\newcommand{\qed}{\begin{center}
$\square$
\end{center}}
\newcommand{\st}{\mid}





%SetFonts

%SetFonts


\title{Selected Exercises \S13}

\author{Colton Kinstley}
%\date{}							% Activate to display a given date or no date

\begin{document}
\maketitle

\section*{Question 1}
\paragraph{}

Let $X$ be a topological space; let $A$ be a subset of $X$. Suppose the for each $x \in A$ there is an open set $U$ containing $x$ such that $U \subset A $. Show that $A$ is open in $X$.


\subsection*{Solution}
\paragraph{}

To show that $A$ is an open set in $X$ we will show that it is the union of open sets of $X$. Given $x \in  A$ let $U_x$ be an open set containing $X$ such that $U \subset A$. Then
\[
\bigcup_{x \in A} U_x = A
\]
thus $A$ is a union of open sets.


\section*{Question 6}

Show that the topologies of $\mathbb{R}_l \text{ and } \mathbb{R}_K$ are not comparable. 

\subsection*{Solution}
\paragraph{}

A basis for $\R_l$ is the set of all intervals $[a,b)$ and a basis for $\R_K$ is the set of all open intervals $(a,b)$ along with the intervals of the form $(a,b) \setminus K$ where $ K = \{1/n \mid n \in \bb{Z_+}\}$. Let $[x,b) $ be a basis element of $\R_l$ where $1< x< b$. Then there is no basis element $B$ of $\R_K$ such that $x \in B \tand B \subset [x,b)$. Thus by lemma 13.3 [Munkres] $\R_K \not\subset \R_l$.

Now take $D = (-1,1) \setminus K$ as a basis element of $\R_K$. No basis element $B$ of $\R_l$ both contains the point 0 and is contained within $D$. If so then $B=(a,b) \twhere a<0<b$ but there exists $n \in \bb{Z}$ such that $0 < 1/n < b$ so $B \not \subset D$. Thus by lemma 13.3 [Munkres] $\R_l \not\subset \R_K$. We conclude that $\R_l \tand \R_K$ are not comparable.


\section*{Question 8}

\subsection*{(a)}
\paragraph{}

ApplyLemma 13.2 to show that the countable collection 

\begin{equation*}
\scr{B} = \{(ab,) \mid a< b, a \tand b \text{ rational} \}
\end{equation*}
is a basis that generated the standard topology on $\R$.


\subsection*{Proof}

Let $\scr{T}$ be the topology generated by $\scr{B}$. We will show that $\scr{T} = \scr{T}_\R$. Each element of $\scr{B}$ is also a basis element of $\scr{T}_\R$ so $ \scr{T} \subset \R$.

 Now suppose $U = (\alpha,\beta)$ is a basis element of $\scr{T}_\R$. For any $ x \in U$ there exists an open interval $(a,b)$ with  $a \tand b$ rational such that $\alpha < a < x < b < \beta$ because $\bb{Q}$ is dense in $\R$. Thus by lemma 13.3 $\scr{T}_\R \subset \scr{T}$. \qed
 
 \subsection*{(b)}
 
 Show that the collection
 
 \begin{equation*}
 \scr{C} = \{[a,b) \mid a < b, a \tand b \text{ rational} \}
 \end{equation*}
 is a basis that generates a topology different from the lower limit topology on $\R$.
 
 \subsection*{Solution}
 \paragraph{}
 
 Let $\scr{T}$ be the topology generated by $\scr{C}$. Take $[x,\beta)$ with $x$ irrational as basis element of $\R_l$. There is no element of $\scr{C}$ that contains $x$ and is contained within $[x,\beta)$. Thus by lemma 13.3 $\scr{T} \not \subset \R_l$. Thus $\scr{T}  \neq \R_l$.
 
 
 \section*{Extra Question}
 \paragraph{}
 
 Determine the convergent sequences for the finite complement topology on $\R$.
 
 \subsection*{Claim}
 \paragraph{}
 
 If ${x_n}$ is a sequence in $(\R, \scr{T}_F)$ then $x_n$ converges if and only if $x_n$ has 0 or 1 constant subsequence.
 
 \subsection*{Proof}
 \paragraph{}
 First we note that any open set $U$ in $(\R, \scr{T}_F)$ can be written in the form 
 \[
U = \R \setminus \{a_i\}_{i=1}^n.
 \]
 
Suppose a sequence $x_n$ has no constant subsequences. We claim that this sequence will converge to any point $l$ in $\R$. Take some neighborhood of $l$, $U = \R \setminus \{a_i\}_{i=1}^n$. Since $x_n$ has no constant subsequences the sequence coincides with $\{a_i\}_{i=1}^n$ on finitely many terms. That is to say $x_{i_k} = a_k$ for only finitely many $i_k \in \bb{N}$. Otherwise, $a_k$ would be the value of a constant subsequence. Thus we may take 

\[
N = \max_{\substack{k \in 1,\dots, n \\ i_k \in \bb{N}}} \{i_k\} + 1.
\]
 in the definition of convergence. So that for all $n \geq N$ we have that $x_n \not \in \{a_i\}_{i=1}^n$ thus $x_n \in U$ for $n \geq N$.
 
 Now suppose that $x_n$ has only one constant subsequence, say $x_m = l$ for $m \in M \subset \bb{N}$ where $M$ has infinite cardinality. Then $x_n$ converges to $l$. To see this take any neighborhood $U$ of $l$. Then $x_m = l \in U$ for all $m \in M$ and the remaining subsequence $x_k \twhere k \in \bb{N} \setminus M$ is a sequence with no constant subsequence. From the above argument it will also converge to $l$, thus $x_n$ as a whole converges to $l$.
 
 For the converse we will prove its contrapositive; if $x_n$ has two or more constant subsequences, then $x_n$ does not converge to any point in $\R$. Suppose $x_n$ is a series with two or more distinct constant subsequences, say
 \begin{eqnarray*}
&x_{m_1} = l_1,\\
&x_{m_2} = l_2, \\
&\vdots \\
&\tfor m_i \in M_i \subset \bb{N} \tand M_i \text{ infinite. }
\end{eqnarray*}
Suppose that $x_n$ does converge to some point $l \in \R$, then $l$ is different from at least one of $l_1$, $l_2, \dots$ and the neighborhood of $l$, $U = \R \setminus \{l_k\}$ where $l \neq l_k$ is such that for all $m_k \in M_k$ we have that  $x_{m_k} \not \in U$. But, since $M_k$ is infinite this  contradicts the supposition that $x_n$ converges to $l$.

 \begin{center}
 $\square$
 \end{center}
 



























\end{document}