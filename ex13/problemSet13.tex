\documentclass[11pt, oneside]{article}   	% use "amsart" instead of "article" for AMSLaTeX format
\usepackage{geometry}                		% See geometry.pdf to learn the layout options. There are lots.
\geometry{letterpaper}                   		% ... or a4paper or a5paper or ... 
%\geometry{landscape}                		% Activate for rotated page geometry
%\usepackage[parfill]{parskip}    		% Activate to begin paragraphs with an empty line rather than an indent
\usepackage{graphicx}				% Use pdf, png, jpg, or eps§ with pdflatex; use eps in DVI mode

								% TeX will automatically convert eps --> pdf in pdflatex		
\usepackage{amssymb}

\usepackage{amsmath}
\usepackage[mathscr]{euscript}


%SetFonts

%SetFonts


\title{Exercises \S13}
\author{Colton Kinstley}
%\date{}							% Activate to display a given date or no date

\begin{document}
\maketitle



%Question 1


\section*{Question 1}
\paragraph{}

Let $X$ be a topological space; let $A$ be a subset of $X$. Suppose the for each $x \in A$ there is an open set $U$ containing $x$ such that $U \subset A $. Show that $A$ is open in $X$.


\subsection*{Solution}
\paragraph{}

To show that $A$ is an open set in $X$ we will show that it is the union of open sets of $X$. Given $x \in  A$ let $U_x$ be an open set containing $X$ such that $U \subset A$. Then
\[
\bigcup_{x \in A} U_x = A
\]
thus $A$ is a union of open sets.


%Question 2

\section*{Question 2}
\paragraph{}

Consider the nine topologies on the set $X = \{a,b,c\}$ indicated in Example 1 of \S12. Compare them; that is, for each pair of topologies, determine whether they are comparable, and if so, which is the finer.

\subsection*{Solution}
\paragraph{}

We shall label the topologies as elements in a matrix and give a sample of comparisons.
\begin{enumerate}
\item $\mathscr{T}_{11}$ is comparable to and coarser than all others.

\item $\mathscr{T}_{33}$ is comparable to and finer than all others.

\item $\mathscr{T}_{12}$, $\mathscr{T}_{31}$ and $\mathscr{T}_{32}$ are comparable with $\mathscr{T}_{31} \subset \mathscr{T}_{12}  \subset  \mathscr{T}_{32}$
\end{enumerate}





%Question 3

\section*{Question 3}
Show that the collection $\mathscr{T}_c$ given in Example  4 of \S12 is a topology on the set $X$. Is the collection

\begin{equation*}
\mathscr{T}_\infty = \{U | X \setminus U \text{ is infinite or empty or all of } X \}
\end{equation*}
a topology on $X$?

\subsection*{Solution (part 1)}
\paragraph{}

Let $\mathscr{T}_c$ be the collection of subsets, $U$ of $X$ such that $X \setminus U = X$ or $X \setminus U$ is countable. First we see that $X \setminus X = \varnothing$ which is countable and $X \setminus \varnothing = X$ so $X, \varnothing \in \mathscr{T}_c$ . 

\paragraph{}
Next we will show that $\mathscr{T}_c$ is closed under unions. Given any union of open sets $\bigcup U_\alpha$ we have that

\[
X \setminus \bigcup U_\alpha = \bigcap (X \setminus U_\alpha).
\]
Because $X \setminus U_\alpha$ is countable for all $\alpha$ and the intersection of countable sets is countable we have that $\bigcup U_\alpha \in \mathscr{T}_c$ .

\paragraph{}
Finally, given any finite intersection $\bigcup_{i=1}^n U_i$ of open sets of $X$ we have that

\[
X \setminus \bigcup_{i=1}^n U_i =  \bigcap_{i=1}^n (U_i \setminus X).
\]
For each $i$, $X \setminus U_i$ is countable and the finite union of countable sets is countable.

\subsection*{Solution (part 2)}
\paragraph{}

$\mathscr{T}_\infty$ is not a topology as it is not closed under finite intersections. For a counterexample consider $X=\mathbb{Z}$ and the two subsets $U_{-1} = \{-1,0, \dots\}$ and $U_1 = \{\dots, 0,1\}$. Though both $U_1$ and $U_{-1}$ are clearly in $\mathscr{T}_\infty$ their intersection,

\[
U_{-1} \bigcap U_1 = \{-1,0,1\}
\]
is finite.

\section*{Question 4a}
\paragraph{}

If $\{\mathscr{T}_\alpha \}$ is a family of topologies on $X$, show that $\bigcap \mathscr{T}_\alpha$ is a topology on $X$. Is $\bigcup \mathscr{T}_\alpha$ a topology on $X$?

\subsection*{Solution (part 1)}
\paragraph{}

We have that $\o$ and $X$ are in $\{\mathscr{T}_\alpha \}$ for all $\alpha$ so $\o, X \in \bigcap \mathscr{T}_\alpha$. Let $\{U_i\}_{i \in I}$ be a collection of sets in $\bigcap \mathscr{T}_\alpha$. Since each $U_i$ is an element of $\mathscr{T}_\alpha$ for all $\alpha$ and $\mathscr{T}_\alpha$ is closed under unions for each $\alpha$ we must have that

\[
\bigcup_{i \in I} U_i \in \bigcap \mathscr{T}_\alpha .
\]
Likewise if $\{U_i\}$ is a finite collection
\[
\bigcap_{i=1}^n U_i \in \mathscr{T}_\alpha
\]
as each $U_i \in \mathscr{T}_\alpha$ for some $\alpha$ and $\mathscr{T}_\alpha$ is closed under finite unions.

\subsection*{Solution (part 2)}
\paragraph{}
No, $\bigcup \mathscr{T}_\alpha$ is not a topology in general. For a counterexample examine the topologies in Example 1 of \S12. Observe that $\mathscr{T}_{12} \bigcup \mathscr{T}_{21}$ is not a topology; for one $\{b\}$ is not in the union but can be obtained from intersection of elements in $\mathscr{T}_{12} \bigcup \mathscr{T}_{21}$.

%Question 4b

\section*{Question 4b}
\paragraph{}

Let $\{ \mathscr{T}_\alpha\}$ be a family of topologies on $X$. Show that there is a unique smallest topology on $X$ containing all the collections $\mathscr{T}_\alpha$, and a unique largest topology contained in all $\mathscr{T}_\alpha$.

\subsection*{Solution}
\paragraph{}

Let $\{ \mathscr{T}_\alpha\}$ be a family of topologies on $X$ and let $\mathscr{A}$ be the collection of all unions and finite intersections of elements in $\bigcup \mathscr{T}_\alpha$. By definition $\mathscr{A}$ is closed under unions and finite intersections and because each topology $\mathscr{T}_\alpha$ contains $X$ and $\varnothing$ so will $\mathscr{A}$; thus $\mathscr{A}$ is a topology.

Clearly for each $\alpha$, we have $\mathscr{T}_\alpha \subset \mathscr{A}$. Suppose there is some other topoogy $\mathscr{A}'$ for which $\mathscr{T}_\alpha \subset \mathscr{A}'$ for all $\alpha$. Given any $U \in \mathscr{A}$ either $U =  \bigcup U_{\alpha}$ or $\bigcap_{i=1}^n U_i$ with $U_\alpha, U_i \in \bigcup \mathscr{T}_\alpha$. However, because $\mathscr{A}'$ is a topology that contains each $\mathscr{T}_\alpha$, we must have that $U \in \mathscr{A}'$. So $\mathscr{A} \subset \mathscr{A}'$.



































\end{document}  